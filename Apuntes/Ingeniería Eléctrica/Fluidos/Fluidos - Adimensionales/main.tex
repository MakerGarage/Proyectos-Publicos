\documentclass[10pt,a4paper]{article}
\usepackage{geometry}
 \geometry{
 a4paper,
 total={170mm,257mm},
 left=20mm,
 top=20mm,
 }


\usepackage[utf8]{inputenc}
\usepackage[spanish]{babel}
\usepackage{graphicx}
\usepackage{wrapfig}
\usepackage{amssymb, amsmath, siunitx}
\usepackage{tikz}
\usepackage{enumitem}
\usepackage{svg}
\usepackage{tcolorbox}
\usepackage{pdfpages}
\usepackage{xcolor}
\usepackage{colortbl}

\setlength{\tabcolsep}{25pt}
\renewcommand{\arraystretch}{1.5}

\title{Adimensionales}
\author{MakerGarage}
\date{Mayo 2021}


\setlength{\parindent}{0cm}

\begin{document}

\maketitle
\newpage
\tableofcontents
\newpage
\setcounter{section}{-1}
\section{Magnitudes expresadas en magnitudes fundamentales}
Las magnitudes fundamentales son masa [kg] longitud [m] y tiempo [s], cualquier otra unidad puede expresarse en base a las unidades fundamentales.

\begin{center}
\begin{tabular}{ c | c | c | c}
  Magnitud  & Simbolo  & Unidades  & Dimensión \\
   \hline
    Longitud & L   & $m$  & L 
    \\\hline
    
    Área & A  & $m^2$  & $L^2$ 
    \\\hline
    
    Volúmen & V  & $m^3$  & $L^3$
    \\\hline
    
    Tiempo & t & $s$  & T 
    \\\hline
    
    Velocidad & U  & $\frac{m}{s}$  & L$T^{-1}$
    \\\hline
    
    Aceleración & a & $\frac{m}{s^2}$ & L$T^{-2}$ 
    \\\hline
    
    Ángulo & $\alpha$ & $^{\circ}$ & 1
    \\\hline
    
   Velocidad angular & $\Omega$ & $\frac{1}{s}$ & $T^{-1}$
   \\\hline
   
    Masa & $m$ & $kg$ & $M$
    \\\hline
    
    Densidad & $\rho$ & $\frac{kg}{m^3}$ & $ML^{-3}$
    \\\hline
    
    Viscosidad & $\mu$ & $\frac{kg}{m s}$ & $M^{}L^{-1}T^{-1}$
    \\\hline
    
    Presión & $p$ & $\frac{kg}{m s^2}$ & $M^{}L^{-1}T^{-2}$
    \\\hline
    
    Fuerza & $F$ & $\frac{kg m}{s^2}$ & $M^{}L^{}T^{-2}$
    \\\hline
    
    Momento / Par & $M$ & $\frac{kg m^2}{s^2}$ & $M^{}L^{2}T^{-2}$
    \\\hline
    
    Potencia & $\dot{W}$ & $\frac{kg m^2}{s^3}$ & $M^{}L^{2}T^{-3}$
    \\\hline
    
    Trabajo / Energía & E & $\frac{kg m^2}{s^2}$ & $M^{}L^{2}T^{-2}$
  
\end{tabular}
\end{center}

 \newpage
\section{Paso a paso}
\subsection{Paso 1 - Identificar las variables del problema}
Normalmente nos dice el enunciado que la potencia, o la velocidad o cualquier otra magnitud depende de otras magnitudes es decir por poner un ejemplo, podemos decir que la Presión depende de la Velocidad, de la masa, de la viscosidad etc...
\\

Por ejemplo:
$$
\dot{W} = \dot{W} \left(L,\rho,\mu,g,m,U \right)
$$

\subsection{Paso 2 - Escribir nuestras magnitudes en función de las fundamentales}
La primera columna corresponde a la magnitud que queremos calcular.
\begin{center}
\begin{tabular}{ c | c | c | c | c | c | c | c}
  & \cellcolor{red!25}$\dot{W}$ & L & U & $\rho$ & $\mu$ & g & m\\
    \hline
    M & 1 & 0 & 0  & 1 & 1 & 0 & 1 \\
    L & 2 & 1 & 1 & -3 & -1 & 1 & 0\\
    T & -3 & 0 & -1 & 0 & -1 & -2 & 0\\
 \end{tabular}
 \end{center}
 

 \subsection{Paso 3 - Seleccionar nuestras variables características}
 Para seleccionar nuestras variables características tenemos que escoger:
 \begin{itemize}
     \item Característica del fluido ($\rho$, $\mu$) 
     \item Característica del flujo (U,Q,$\Omega$)
     \item Característica geométrica (L,D)
 \end{itemize}
 Fluido $\xrightarrow{}$ Lo que va a fluir (agua, aceite etc)
 \\
 Flujo $\xrightarrow{}$ Como fluye (velocidad, temperatura etc)
 \\
 Característica geométrica $\xrightarrow{}$ Por donde fluye (Longitud, diámetro etc)
 \\
 
 \begin{center}
\begin{tabular}{ c | c | c | c | c | c | c | c}
  & \cellcolor{red!25}$\dot{W}$ & \cellcolor{blue!25}L & \cellcolor{blue!25}U & \cellcolor{blue!25}$\rho$ & $\mu$ & g & m\\
    \hline
    M & 1 & 0 & 0  & 1 & 1 & 0 & 1 \\
    L & 2 & 1 & 1 & -3 & -1 & 1 & 0\\
    T & -3 & 0 & -1 & 0 & -1 & -2 & 0\\
 \end{tabular}
 \end{center}
 
 \subsection{Paso 4 - Verificar que son linealmente independientes}
 Tenemos que hacer el determinante y verificar que es distinto de 0. En caso de no serlo tenemos que coger otra variable para que sean linealmente independientes.
 \begin{equation}
\begin{vmatrix}
0 &  0 & 1\\
1 & 1 & -3\\
0 & -1 & 0
\end{vmatrix}
= -1 \not = 0
\end{equation}

\subsection{Paso 5 - Calcular el Teorema de PI}
El teorema de pi nos permite saber cuantos grupos pi tenemos que calcular
n = $m - k$
\\

m = numero de variables (6 en nuestro caso)
\\
k = rango de la matriz (3 en nuestro caso)

\subsection{Paso 6 - Crear los grupos PI}
Siempre el primer grupo pi $pi_0$ corresponde a la magnitud a estudiar en nuestro caso $\dot{W}$ y se relaciona con las magnitudes que hemos seleccionado (azul)
\\

Ejemplo general para $\dot{W}$ y para L.
\\
\vspace{0.2cm}
$\pi_0$ = $\dot{W} \cdot \left[L \right]^a \cdot \left[U \right]^b \cdot \left[\rho \right]^c$
\\
\vspace{0.2cm}
$\pi_1$ = L $\cdot \left[L \right]^a \cdot \left[U \right]^b \cdot \left[\rho \right]^c$

\subsection{Paso 7 - Resolver cada grupo PI}
Vamos a resolver el grupo $\pi_0$
$$
\pi_0 = \dot{W} \cdot \left[L \right]^a \cdot \left[U \right]^b \cdot \left[\rho \right]^c
$$

$$
M^{0}L^{0}T^{0} = M^{}L^{2}T^{-3} \cdot \left[M^{0}L^{1}T^{0} \right]^a \cdot \left[M^{0}L^{}T^{-1} \right]^b \cdot \left[M^{1}L^{-3}T^{0} \right]^c
$$
\vspace{0.3cm}
Igualamos M con M L con L y T con T.

$
\left.
\begin{array}{rcl}
    0 & = & 1+c\\
    0 & = & 2+a+b-3c \\
    0 & = & -3-b
\end{array}
\right\}
\xrightarrow{}
\begin{array}{rcl}
    a = -3 \\
    b = -2 \\ 
    c = -1
\end{array}
$
Sustituimos los valores de a b y c.
$$
\pi_0 = \dot{W} \cdot \left[L \right]^{-3} \cdot \left[U \right]^{-2} \cdot \left[\rho \right]^{-1}
$$

$$
\pi_0 = \frac{\dot{W}}{L^3 U^2 \rho}
$$

\textbf{Importante}:
\\

\underline{Reynolds}
$$
\pi_x = \frac{V \rho D}{\mu} = \text{Re}
$$
$$
\pi_x = \frac{\mu}{V \rho D} = \frac{1}{\text{Re}} = Re
$$

\underline{Froude}
$$
\pi_x = \frac{U^2}{g L} = \text{Fr}
$$
$$
\pi_x = \frac{g L}{U^2} = \frac{1}{\text{Fr}} = Fr
$$

\subsection{Paso 8 - Resolver el ejercicio}
Una vez tenemos todos los grupos PI necesarios procedemos a resolver:

$$
\pi_0 = f\left(\pi_1,\pi_2,\pi_3 \right)
$$

$$
\frac{\dot{W}}{L^3 U^2 \rho}= f\left(Re,Fr,\frac{m}{\rho L^3}\right)
$$

\end{document}
